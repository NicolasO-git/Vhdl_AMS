\section{Analyse instanciation véhicule}
Pour cette première partie nous avons isntancié le véhicule représenté 
\begin{center}
    \begin{tabular}{|l|c|r|}
      \hline
      Courant (mA) & Tension (mV) & Puissance (mW) \\
      \hline
      0 & 0 & 0 \\
      3 & 0 & 0 \\
      6 & 28 & 0.0403 \\
      9 & 142 & 0.2046 \\
      12 & 249 & 0.3589 \\
      15 & 355 & 0.511 \\
      18 & 462 & 0.665 \\
      \hlc[yellow]{20} & \hlc[yellow]{555} & \hlc[yellow]{0.8} \\
      \hline
    \end{tabular}
\end{center}
Pour convertir l'échelle de tension en échelle de puissance nous avons appliqué : \[P_{opt} = \frac{P_{ref} \times T_{sortie}}{T_{ref}}\]
\newpage